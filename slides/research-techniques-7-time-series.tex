%%----------------------------------------------------------------------------
%% Presentatie HoGent Bedrijf en Organisatie
%%----------------------------------------------------------------------------
%% Auteur: Bert Van Vreckem [bert.vanvreckem@hogent.be]

\documentclass{beamer}

%==============================================================================
% Aanloop
%==============================================================================

%---------- Packages ----------------------------------------------------------
\usepackage{etex}
\usepackage{graphicx,multicol}
\usepackage{comment,enumerate,hyperref}
\usepackage{amsmath,amsfonts,amssymb}
\usepackage{tikz}
\usepackage[english]{babel}
\usepackage[utf8]{inputenc}
\usepackage{multirow}
\usepackage{eurosym}
\usepackage{listings}
\usepackage[T1]{fontenc}
\usepackage{lmodern}
\usepackage{textcomp}
\usepackage{framed}
\usepackage{wrapfig}
\usepackage{pgf-pie}
\usepackage{pgfplots}
\usepackage{booktabs}
\usepackage{pgfplotstable}
\usepackage{changepage}
\usepackage{pst-plot,pst-func}

%---------- Configuratie ------------------------------------------------------

\usetikzlibrary{arrows,shapes,backgrounds,positioning,shadows}
\usetikzlibrary{pgfplots.statistics}
\newif\ifprivate
\privatetrue


\usetheme{hogent}
\setbeameroption{show notes}

%---------- Commando-definities -----------------------------------------------

\newcommand{\tabitem}{~~\llap{\textbullet}~~}
\renewcommand{\arraystretch}{1.2}

%---------- Info over de presentatie ------------------------------------------

\title[Intro]{Research techniques\\Time series}
\author{Wim Goedertier, Jens Buyse, Bert {Van Vreckem}, Wim {De Bruyn}}
\date{AY 2017-2018}

%==============================================================================
% Inhoud presentatie
%==============================================================================

\begin{document}

%---------- Front matter ------------------------------------------------------

% Dia met het HoGent logo
\HoGentLogo

% Titeldia met faculteitslogo
\titleframe

%---------- Inhoud ------------------------------------------------------------

\begin{frame}
  \frametitle{What's on the menu today?}

  \tableofcontents
\end{frame}

\section{Time series and predictions}

\begin{frame}
  \frametitle{Time series and predictions}

  \brightbox{A \textcolor{HoGentYellow}{time series} is a sequence of observations of a variable over time}
\vfill
  A time series is a stochastic process. Examples include:

  \begin{itemize}
    \item Monthly demand for milk
    \item Yearly influx of ``generational students'' in a college
    \item The price of a stock or bond at the stock exchange (day to day, hour to hour, etc.)
    \item The number of HTTP requests per second on a website, or the response time for those requests
    \item Evolution of disk usage on a backup server
  \end{itemize}
\end{frame}

\begin{frame}
  \frametitle{Time series and predictions}
  
  Time series are an important part of research, because the often are the \emph{foundation} of decision models and predictions.
  
  \begin{itemize}
    \item Capacity planning (e.g. disk storage, computing power)
    \item Reaching financial objectives, a.o.~investments
    \item Planning operational/marketing budgets
    \item \dots
  \end{itemize}
\end{frame}

\begin{frame}
  \frametitle{Time series and predictions}

  Time series are a \emph{statistical} problem: observations will vary over time.
  
  \begin{figure}
    \centering
    \includegraphics[width=\textwidth]{img/tijdreeks11}
    \caption{Time series for the weekly demand for some product.}
  \end{figure}
\end{frame}

\section{Time series models}

\subsection{Mathematical model}

\begin{frame}
  \frametitle{Mathematical models for time series}

  \begin{enumerate}
  \item The \textbf{constant} model
\vfill
The observed values $x_t$ are modelled as random variations $\varepsilon_t$ around a constant value $b$

\begin{equation}
x_{t} = b + \varepsilon_{t}
\label{eq:timeseries-constant}
\end{equation}

  \begin{itemize}
    \item $x_{t}$: \emph{observation} at time $t$
    \item $b$: the constant, i.e. the average of the observations $x_t$
    \item $\varepsilon_{t}$: random variation, \textbf{noise}. We assume $\varepsilon_{t} \sim Nor(\mu = 0; \sigma)$
  \end{itemize}
  \end{enumerate}
\end{frame}

\begin{frame}
  \frametitle{Mathematical models for time series}

  \begin{enumerate}
  \setcounter{enumi}{1}
  \item The \textbf{linear} model
  \begin{equation}
    x_{t} = b_{0} + b_{1} \cdot t + \varepsilon_{t}
    \label{eq:timeseries-linear}
  \end{equation}
\vfill
  \item The \textbf{polynomial} model
  \begin{equation}
    x_{t} = b_{0} + b_{1} t + b_{2} t^{2} + \dots + b_{n} t^{n} + \varepsilon_{t}
    \label{eq:timeseries-polynomial}
  \end{equation}
\vfill
\item The \textbf{general} parametric model
  \begin{equation}
    x_{t} = f(b_{0}, b_{1}, b_{2}, \ldots , b_{n}, t) + \varepsilon_{t}
    \label{eq:timeseries-parametric}
  \end{equation}
$f(\ldots,t)$ can be any function with parameters $b_i$
  \end{enumerate}
\end{frame}

\subsection{Parameter estimation}

\begin{frame}
  \frametitle{Parameter estimation}

  Goal = make predictions (to do \textit{forecasting})

  \begin{enumerate}
    \item select the most appropriate model
    \item estimate the parameters $b_i$ ($i=1, \dots, n$) based on the observations $x_t$
  \end{enumerate}

  These estimates $\widehat{b_i}$ are chosen in a way that they approach the observations as well as possible.
\vfill
  Mathematically, we need to assume \ldots
\begin{itemize}
    \item \ldots that the variability on $x_t$ can be split in 2 components:
    \begin{itemize}
        \item an average that varies over time
        \item a \textbf{random} variation on top of this average ($\varepsilon_t$)
    \end{itemize}
    \item \ldots that the variance of $\varepsilon_{t}$ is \textbf{constant} over time\\
    (i.e.: $x_t$ is \textbf{homoscedastic})
\end{itemize}

\end{frame}

\begin{frame}<presentation:0> %hide this slide
  \frametitle{Parameter estimation: example}

  \begin{table}[t]
    \centering
    \begin{tabular}{|l|l|l|l|l|l|l|l|l|l|}
      \hline
      4 & 16 & 12 & 25 & 13 & 12 & 4 & 8  & 9 & 14 \\ \hline
      3 & 14 & 14 & 20 & 7  & 9  & 6 & 11 & 3 & 11 \\ \hline
      8 & 7  & 2  & 8  & 8  & 10 & 7 & 16 & 9 & 4  \\ \hline
    \end{tabular}
    \label{tab:data}
    \caption{Time series of the weekly demand for some product}
  \end{table}

  \begin{figure}
    \centering
    \includegraphics[width=.7\textwidth]{img/tijdreeks11}
  \end{figure}
\end{frame}

\begin{frame}<presentation:0> %hide this slide
  \frametitle{Parameter estimation: example}

  \begin{itemize}
    \item We choose the constant model from Equation~\ref{eq:timeseries-constant}
    \item As estimate for $b$, we choose the average of the first 20 observations:

      \[ \widehat{b} = \frac{1}{20} \sum_{t = 1}^{20} x_{t}= 10.75 \]

  \end{itemize}

  \centering
  \includegraphics[width=.7\textwidth]{img/tijdreeks21.jpg}
\end{frame}

\begin{frame}%<presentation:0> %hide this slide
  \frametitle{Parameter estimation: example}

  The observations used for the estimate of $\widehat{b}$ can be chosen, e.g.:

  \begin{itemize}
    \item $\widehat{b} = \frac{1}{10} \sum_{10}^{20} x_{t} = 10.18$
    \item $\widehat{b} = \frac{1}{5} \sum_{15}^{20} x_{t} = 7.83$
  \end{itemize}

\end{frame}

\section{Moving average}

\subsection{Simple moving average (SMA)}

\begin{frame}
  \frametitle{Simple Moving average}

  \brightbox{The \textcolor{HoGentYellow}{Simple Moving Average} is a series of \emph{averages} of the last $m$ observations}

  \begin{itemize}
    \item The parameter $m$ is called the \textbf{time range}
    \item Used to hide short term fluctuations
    \item Used to show long term trends
  \end{itemize}

  \begin{equation}
    \textnormal{SMA}_{m,t} = \sum_{i=k}^{t} \frac{x_{i}}{m}
    \label{eq:movingAverage}
  \end{equation}

  with $k = t - m + 1$.
\end{frame}

\begin{frame}
  \frametitle{Application: ``Golden cross''}

  Moving Averages are used in the \emph{technical analysis} of stock prices in order to discover trends.

  \begin{center}
    \includegraphics[width=\textwidth]{img/tijdreeks-golden-cross}
  \end{center}
\end{frame}

\note{%
  This chart shows the evolution of the S\&P500 index (compare to Bel-20, Euro Stoxx 50) from December 2011 to the end of 2012. The chart shows the SMA's of the price at closing time of the 50 and 200 last trade days (notated: MA(50) and MA(200)). When the market is in a downward trend (``\textbf{bear market}''), the MA(200) is above MA(50) (see left part of the chart).

  In February 2012, the MA(50) rose above the MA(200), an event that is called a ``\textbf{golden cross}''. The increase of the index had started a few months earlier, but the MA's obviously are slower.
  
  A \emph{golden cross} is an indicator that the market (or a specific traded equity) is in a long term upward trend (``\textbf{bull market}''). In this case, the trend lasts up to the present day (spring of 2017).

  The MA(200)-line is considerd to be a support, i.e.~a lower bound for the stock price. At this level, the stock price is appealing, so demand typically rises and the price is pulled up again. That is, if the circumstances haven't changed (the stock's \emph{fundamentals}, economy in general, \dots).

}

\begin{frame}<presentation:0> %hide this slide
  \begin{table}
    \begin{tabular}{|llllllllll|}
      \hline
      $t$       & 11   & 12   & 13   & 14   & 15   & 16   & 17   & 18   & 19   \\
      $x_{t}$ & 3    & 14   & 14   & 20   & 7    & 9    & 6    & 11   & 3    \\
      SMA$_{10,t}$ & 11.7 & 11.6 & 11.4 & 11.6 & 11.1 & 10.5 & 10.2 & 10.4 & 10.7 \\
      $e$     & -8.7 & 2.4  & 2.6  & 8.4  & -4.1 & -1.5 & -4.2 & 0.6  & -7.7 \\ \hline
    \end{tabular}
    \caption{Error in moving average ($m = 10$) predictions.}
    \label{tab:error}
\end{table}

A method to measure the quality of the predictions is the mean absolute devations (MAD):

\begin{equation}
  MAD = \frac{1}{n} \sum_{1}^{n} \left| e_{i} \right|
\label{eq:MAD}
\end{equation}

You can also calculate the mean squared deviation/mean squared error (MSE):

\begin{equation}
  s^{2}_{e} = \frac{1}{m} \sum_{1}^{n} (e_{i} - \overline{e})^{2}
\label{eq:varError}
\end{equation}
\end{frame}

\subsection{Weighted Moving Average}

\begin{frame}
  \frametitle{Weighted Moving Average}

  \begin{itemize}
    \item In an \textbf{SMA} all observations contribute equally
    \[ \textnormal{Eg.: SMA}_{4,t} = \frac{1}{4} \cdot x_{t} + \frac{1}{4} \cdot x_{t-1} +
\frac{1}{4} \cdot x_{t-2} + \frac{1}{4} \cdot x_{t-3} \]
\vfill
    \item In a \textit{Weighted Moving Average} (\textbf{WMA}) the observations have different weights (multiplying factors)
    \[ \textnormal{Eg.: WMA}_t = \frac{4}{10} \cdot x_{t} + \frac{3}{10} \cdot x_{t-1} +
\frac{2}{10} \cdot x_{t-2} + \frac{1}{10} \cdot x_{t-3} \]
  \end{itemize}
\end{frame}

\begin{frame}
  \frametitle{Exponential Moving Average}
The \emph{Exponential Moving Average} (\textbf{EMA}) or \textbf{exponential smoothing} is a special case of WMA:
\begin{equation}
\textnormal{EMA}_t = S_t = \alpha \cdot x_t + (1-\alpha) \cdot S_{t-1}
\label{eq:singleExpMA}
\end{equation}

The parameter $\alpha$ is called the \textbf{smoothing constant} ($0 < \alpha < 1$). 
\vfill
Formula~(\ref{eq:singleExpMA}) is only valid for $t \geq 2$.\\
It is important to choose a good value for $S_1$, possible choices:
  
  \begin{itemize}
    \item $S_1 = x_1$
    \item $S_1 = \frac{1}{m} \sum_{i=1}^{m} x_i$ (i.e. the mean of the first $m$ observations)
    \item \ldots
  \end{itemize}
\end{frame}

\begin{frame}
  \frametitle{Why ``\emph{exponential}''?}

$ S_2 = \alpha \cdot x_2 + (1-\alpha) \cdot S_1 $\\
\vfill
$ S_3 = \alpha \cdot x_3 + (1-\alpha) \cdot S_2 $\\
$ S_3 = \alpha \cdot x_3 + (1-\alpha) \cdot \left[\alpha \cdot x_2 + (1-\alpha) \cdot S_1\right]  $\\
$ S_3 = \alpha \cdot x_3 + \alpha (1-\alpha) \cdot x_2 + (1-\alpha)^2 \cdot S_1  $\\
\vfill
$ S_4 = \alpha \cdot x_4 + (1-\alpha) \cdot S_3 $\\
$ S_4 = \alpha \cdot x_3 + (1-\alpha) \cdot \left[ \alpha \cdot x_3 + \alpha (1-\alpha) \cdot x_2 + (1-\alpha)^2 \cdot S_1\right]  $\\
$ S_4 = \alpha \cdot x_4 + \alpha (1-\alpha) \cdot x_3 + \alpha (1-\alpha)^2 \cdot x_2 + (1-\alpha)^3 \cdot S_1  $\\
\vfill
$ S_5 = \alpha \cdot x_5 + \alpha (1-\alpha) \cdot x_4 + \alpha (1-\alpha)^2 \cdot x_3 + \alpha (1-\alpha)^3 \cdot x_2 + (1-\alpha)^4 \cdot S_1  $\\
\vfill
%$ S_k = \alpha \cdot x_k + \alpha (1-\alpha)^1 \cdot x_{k-1} + \ldots + \alpha (1-\alpha)^{k-2} \cdot x_2 + (1-\alpha)^{k-1} \cdot S_1  $\\
%\vfill
The weight factors for older observations decrease \textit{exponentially}
\end{frame}

\begin{frame}
  \frametitle{Exponential smoothing}

  \begin{table}
    \centering
    \begin{tabular}{l|l|l|l|l}
      $\alpha$ & $\alpha(1-\alpha)$ & $\alpha(1-\alpha)^{2}$ & $\alpha(1-\alpha)^{3}$ & $(1-\alpha)^{4}$ \\ \hline
      0.9   & 0.09       & 0.009             & 0.0009                      & 0.0001           \\
      0.5   & 0.25       & 0.125             & 0.0625                      & 0.0625            \\
      0.1   & 0.09       & 0.081             & 0.0729                      & 0.6561           \\
    \end{tabular}
    \caption{Values for $\alpha$ and $(1-\alpha)^{n}$}
    \label{tab:alpha}
  \end{table}
  How quickly old observations are ``forgotten'' depends on $\alpha$.\\
  With $\alpha$ close to 1, old values are forgotten quickly\\
  With $\alpha$ close to 0, older values are ``remembered'' longer
\end{frame}


\begin{frame}
  \frametitle{Example}
  \begin{figure}[htbp]
    \centering
    \includegraphics[width=\textwidth]{img/tijdreeks51}
    \caption{Exponential moving averages with $\alpha=0.1 , 0.5, 0.9$}
    \label{fig:tijdreeks51}
  \end{figure}
\end{frame}

\subsection{Double exponential smoothing}

\begin{frame}
  \frametitle{Double exponential smoothing}

  When there's a trend in the data, exponential smoothing doesn't work very well
  
  \begin{figure}
    \centering
    \includegraphics[width=.7\textwidth]{img/tijdreeks61}
    \caption{Exponential smoothing with a trend: errors increase}
    \label{fig:tijdreeks61}
  \end{figure}

\end{frame}


\begin{frame}
  \frametitle{Double exponential smoothing}
  
  We introduce an extra term in the model to estimate the trend: $b_t$ at time $t > 1$:
\begin{eqnarray}
  S_{t} = \alpha \cdot x_{t} + (1-\alpha) \cdot (S_{t-1} + b_{t-1}) & ~~~0 < \alpha < 1 \\
  b_{t} = \beta \cdot (S_{t}-S_{t-1}) + (1-\beta) \cdot b_{t-1}     & ~~~0 < \beta < 1 
\label{eq:doubleSmoothing}
\end{eqnarray}
\vfill
Remark:
  \begin{itemize}
    \item $ b_{t-1}$ in the first equation ensures estimates follow the trend
    \item $S_{t}-S_{t-1}$ is used to update the trend estimate. It can be positive (rising trend) or negative (downward trend)
  \end{itemize}
\end{frame}

\begin{frame}
  \frametitle{Double exponential smoothing}
  
  Possible choices for bootstrapping:
  
\begin{align*}
  S_{1} & = x_{1} \\
  \\
  b_{1} & = x_{2} - x_{1} \\
  b_{1} & = \frac{ (x_{4} - x_{3}) + (x_{3} - x_{2}) + (x_{2} - x_{1}) }{3} = \frac{x_4-x_1}{3} \\
  b_{1} & = \frac{x_{n} - x_{1}}{n-1} \\
\end{align*}

\end{frame}

\subsubsection{Forecasting}

\begin{frame}
  \frametitle{Double exponential smoothing}
  \framesubtitle{Forecasting}
  
  To determine a forecast $F_{t+1}$ for time index $t+1$, we use:

  \[ F_{t+1} = S_t + b_t \]

  or in general, for time index $t+m$:

  \[ F_{t+m} = S_t + m \cdot b_t \]
\end{frame}

\begin{frame}
  \begin{figure}
    \centering
    \includegraphics[width=\textwidth]{img/tijdreeks71}
    \caption{Basic and double exponential smoothing}
    \label{fig:tijdreeks71}
  \end{figure}
\end{frame}

\subsection{Triple exponential smoothing}

\begin{frame}
  \frametitle{Triple exponential smoothing}
  also called Holt-Winters filtering. Also takes seasonality into account.

  \begin{itemize}
    \item $L$: length of the seasonal cycle (number of time units)
    \item $c_t$: term modeling the seasonal variations
    \item $\gamma$: smoothing factor for the seasonal variation
  \end{itemize}

\begin{align*}
  S_{t} &= \alpha \cdot \frac{x_{t}}{c_{t-L}} + (1-\alpha) \cdot (S_{t-1} + b_{t-1}) & \textnormal{Smoothing}\\
  b_{t} &= \beta \cdot (S_{t} - S_{t-1}) + (1-\beta) \cdot b_{t-1} & \textnormal{Trend smoothing} \\
  c_{t} &= \gamma \cdot \frac{x_{t}}{S_{t}} + (1-\gamma) \cdot c_{t-L} & \textnormal{Seasonal smoothing} \\
\label{eq:HoltWinters}
\end{align*}

\end{frame}

\begin{frame}
  \frametitle{Triple exponential smoothing}
  \framesubtitle{Forecasting}
  Forecast:
  \[ F_{t+m} = (S_{t} + m \cdot b_{t}) \cdot c_{t-L+m} \]
\vfill  
  For an implementation in Java, see \url{https://github.com/bertvv/wintersmethod}\\
\end{frame}

\begin{frame}
  \frametitle{Example: sales forecast}

  \centering
  \begin{tikzpicture}
    \begin{axis}[
        title=Observed sales figures,
        xlabel=Weekday,
        ylabel=SKUs,
      ]
      \addplot table [x index=0, y index=1, col sep=comma] {data/shoestore-sales.dat};
    \end{axis}
  \end{tikzpicture}
\end{frame}

\begin{frame}
  \frametitle{Example: sales forecast}

  Keuze startwaarden:

  \begin{itemize}
    \item smoothing factors: $\alpha = 0.8, \beta = 0.8, \gamma = 0.3$
    \item $s_0 = 5849$, $b_0 = 123.3$
    \item $L = 7$ (i.e.~weekly repetition), so we need 7 values to initialise $c_t$ (see Table~\ref{tab:winters-init-c})
  \end{itemize}

  \begin{table}
    \centering
    \begin{tabular}{l|l|l|l}
      Mon ($c_0$) & Tue ($c_1$) & Wed ($c_2$) & Thu ($c_3$)  \\
      1.245693 & 1.115265 & 1.088853 & 1.135378 \\
      \hline \hline
      Fri ($c_4$)  & Sat ($c_5$)  & Sun ($c_6$)  & \\
      1.178552 & 1.229739 & 0.006520 &
    \end{tabular}
    \caption{Initial values for $c_t$}
    \label{tab:winters-init-c}
  \end{table}
\end{frame}

\begin{frame}
  \frametitle{Example: sales forecast}


  \begin{center}
  \begin{tikzpicture}
    \begin{axis}[
        scale=.8,
        title=Sales figures,
        xlabel=Weekday,
        ylabel=SKUs,
        unbounded coords=jump,
        legend to name=bottomlegend,
      ]
      \addplot table [x index=0, y index=1, col sep=comma] {data/shoestore-sales.dat};
      \addlegendentry{Observed}
      \addplot table [x index=0, y index=2, col sep=comma] {data/shoestore-sales.dat};
      \addlegendentry{Predicted}
    \end{axis}
  \end{tikzpicture}

  \ref{bottomlegend}
\end{center}

\end{frame}


\end{document}

