%%----------------------------------------------------------------------------
%% Presentatie HoGent Bedrijf en Organisatie
%%----------------------------------------------------------------------------
%% Auteur: Bert Van Vreckem [bert.vanvreckem@hogent.be]

\documentclass{beamer}

%==============================================================================
% Aanloop
%==============================================================================

%---------- Packages ----------------------------------------------------------
\usepackage{etex}
\usepackage{graphicx,multicol}
\usepackage{comment,enumerate,hyperref}
\usepackage{amsmath,amsfonts,amssymb}
\usepackage{tikz}
\usepackage[dutch]{babel}
\usepackage[utf8]{inputenc}
\usepackage{multirow}
\usepackage{eurosym}
\usepackage{listings}
\usepackage[T1]{fontenc}
\usepackage{lmodern}
\usepackage{textcomp}
\usepackage{framed}
\usepackage{wrapfig}
\usepackage{pgf-pie}
\usepackage{pgfplots}
\usepackage{booktabs}
\usepackage{pgfplotstable}
\usepackage{changepage}
\usepackage{ulem} % for \sout{text} (strikethrough)
\usepackage{fancyvrb} % for \begin{Verbatim} (LaTeX controls within verbatim)
    
%---------- Configuratie ------------------------------------------------------

\usetikzlibrary{arrows,shapes,backgrounds,positioning,shadows}
\usetikzlibrary{pgfplots.statistics}


\usetheme{hogent}
\setbeameroption{show notes}

%---------- Commando-definities -----------------------------------------------

\newcommand{\tabitem}{~~\llap{\textbullet}~~}
\renewcommand{\arraystretch}{1.2}

%---------- Info over de presentatie ------------------------------------------

\title[Hypothesis tests]{Research techniques\\Hypothesis tests}
\author{Wim Goedertier, Jens Buyse, Bert {Van Vreckem}, Wim {De Bruyn}}
\date{AY 2017-2018}

%==============================================================================
% Inhoud presentatie
%==============================================================================

\begin{document}

%---------- Front matter ------------------------------------------------------

% Dia met het HoGent logo
\HoGentLogo

% Titeldia met faculteitslogo
\titleframe

%---------- Inhoud ------------------------------------------------------------

\pgfmathdeclarefunction{gauss}{2}{%
  \pgfmathparse{1/(#2*sqrt(2*pi))*exp(-((x-#1)^2)/(2*#2^2))}%
}

\begin{frame}
  \frametitle{What's on the menu today?}

  \tableofcontents
\end{frame}

\section{Statistical hypothesis testing}
\sectionframelogo{}

\begin{frame}
  \frametitle{Statistical hypothesis testing}

  \begin{description}
    \item[Hypothesis] A statement about the numerical value of a population parameter that still needs to be proven.
    \item[Hypothesis test] Procedure to validate a hypothesis on one or more population parameters
  \end{description}
\end{frame}

\begin{frame}
  \frametitle{Elements of hypothesis testing}

  \begin{description}
    \item[Null hypothesis ($H_0$)] a hypothesis that we'll be trying to invalidate by a proof by contradiction.
    \item[Alternative hypothesis ($H_1$, $H_a$)] The actual hypothesis we want to prove
    \item[Test statistic] A value that is calculated from the sample
    \item[Region of acceptance] The range of values of the test statistic that \textit{support} the null hypothesis
    \item[Critical region] The range of values of the test statistic for which we can \textit{reject} the null hypothesis
    \item[Significance level] The maximum probability the researcher is willing to accept that $H_0$ is \textit{falsely} rejected
  \end{description}
\end{frame}

\section{Procedure}
\sectionframelogo{}

\begin{frame}
  \frametitle{Procedure}

  \begin{enumerate}
    \item Determine the hypotheses ($H_0$ and $H_1$)
    \item Define the significance level ($\alpha$)
    \item Calculate the test statistic
    \item Determine the region of acceptance and the critical region
    \item Draw conclusions
  \end{enumerate}
\end{frame}

\begin{frame}
  \frametitle{Hypotheses about super heroes}

  \includegraphics[width=\textwidth]{img/les5-heroes}
\end{frame}

\begin{frame}
  \frametitle{A super hero saves more than 3.3 people per day}

  \begin{columns}
    \column{.5\textwidth}
    \centering
    \includegraphics[width=4cm]{img/les5-gered1}

    \includegraphics[width=4cm]{img/les5-gered2}
    \column{.5\textwidth}
    \centering
    \includegraphics[width=4cm]{img/les5-gered3}

    \includegraphics[width=4cm]{img/les5-gered4}
  \end{columns}

  \vfill
  \centering
  \small{Bron: \url{http://www.cracked.com/quick-fixes/4-people-who-saved-day-while-dressed-as-superheroes/}}
\end{frame}

\begin{frame}
Assume, during a period of 30 days, on average, there were 3.483 people saved per day ($\overline{x}=3.483$, $n=30$)
\vfill
  \begin{enumerate}
    \item Hypothesis: $H_0: \mu = 3.3$; $H_1: \mu > 3.3$
    \item Significance level: $\alpha = 0.05$
    \item Sample statistic: $\overline{x} = 3.483$
  \end{enumerate}
\vfill
Population standard deviation (assumed to be known): $\sigma = 0.55$
\end{frame}

\begin{frame}
  \frametitle{Calculate test statistic}
  
  From the central limit theorem follows: $M \sim Nor(\mu = 3.3; \sigma = \frac{0.55}{\sqrt{30}}=0.1)$

  \centering
  \begin{tikzpicture}
    \begin{axis}[domain=3.0:3.6, ymax=4.2, samples=100, enlargelimits=false ]
      \addplot [very thick,smooth,draw=HoGentFBO] {gauss(3.3, 0.10041580220928045413)};
      \node [coordinate, pin={$\overline{x}= 3.483$}] at (axis cs: 3.483, 0){};
    \end{axis}
  \end{tikzpicture}
\end{frame}

\section{\texorpdfstring{$p$}{p}-value}
\sectionframelogo{}

\begin{frame}
  \frametitle{$p$-value}
  \brightbox{The \textcolor{HoGentAccent6}{$p$-value} is the probability, given the null hypothesis is true, the test statistic has a value that is at least as extreme as the observed value.}

  \begin{itemize}
    \item $p$-value $< \alpha \Rightarrow$ reject $H_{0}$: the observed value for $\overline{x}$ is too extreme to be explained by chance.
    \item $p$-value $\geq \alpha \Rightarrow$ don't reject $H_{0}$: the observed value for $\overline{x}$ can be explained by chance.
  \end{itemize}
\end{frame}

\begin{frame}
  \frametitle{$p$-value}

  \centering
  \begin{tikzpicture}[scale=0.8]
    \begin{axis}[domain=-3.5:3.5, ymax=0.42, samples=100, enlargelimits=false, clip=false ]
      \addplot [smooth, fill=black!20, domain=1.822:3.5] {gauss(0,1)} \closedcycle;
      \addplot [very thick,smooth,draw=HoGentFBO] {gauss(0, 1)};
      \node  at (axis cs: 1.822, -.04){\small 1.822};
      \draw [-latex](axis cs:2.1,0.02) -- (axis cs:2.35,0.06);
      \node [text width=0.5cm] at (axis cs: 2.5, .095){$p= 0.0344$};
    \end{axis}
  \end{tikzpicture}
  
  \[ P(M > 3.483) = P \left(Z> \frac{3.483 - 3,3}{\frac{\sigma}{\sqrt{n}}}\right) = P (Z > 1.822) = 0.0344 \]
\end{frame}

\section{Critical region}
\sectionframelogo{}

\begin{frame}
  \frametitle{Critical region}
  
  \brightbox{The \textcolor{HoGentAccent6}{critical region} is the set of values of the test statistic for which we can reject the null hypothesis.}
  
  We are looking for a \emph{critical value} $g$ for which:
\[ P(M > g) = \alpha \]

Find a $z_\alpha$ for which:
\[ P(Z > z_\alpha) = \alpha \]
\[ \Rightarrow g = \mu + z_\alpha \cdot \frac{\sigma}{\sqrt{n}} \]
 
  \begin{itemize}
    \item $\overline{x}<g$: region of acceptance (don't reject $H_0$)
    \item $\overline{x}>g$: critical region (reject $H_0$)
  \end{itemize}
  
\end{frame}


\begin{frame}
  \frametitle{Critical region normal distribution}

  \centering
  \begin{tikzpicture}
    \begin{axis}[domain=-3.5:3.5, ymax=0.42, samples=100, enlargelimits=false, clip=false ]
    \draw [-](axis cs:1.645,-0.03) -- (axis cs:1.645,0.06);
    \addplot [smooth, fill=black!20, domain=1.645:3.5] {gauss(0,1)} \closedcycle;
    \addplot [very thick,smooth,draw=HoGentFBO] {gauss(0, 1)};
    \node [text width=0.8cm] at (axis cs: 1.645, -.06){$z_{\alpha}= 1.645$};
    \draw [-latex](axis cs:2.1,0.02) -- (axis cs:2.35,0.06);
    \node [text width=0.5cm] at (axis cs: 2.5, .095){$\alpha= 0.05$};
    \end{axis}
  \end{tikzpicture}
  
significance level $\alpha = 0.05$ $\Rightarrow$ critical value $z_{\alpha}$=1.645\\
($z_{\alpha} = $\texttt{qnorm($1-\alpha$)=qnorm(0.95)})
\end{frame}

\begin{frame}
  \frametitle{Left tailed test}
  In case of a \emph{left tailed} test, the critical value $g$ becomes:
  \[ g = \mu - z_\alpha \cdot \frac{\sigma}{\sqrt{n}} \]
  and $H_0$ will be rejected if $\overline{x}<g$
\end{frame}

\begin{frame}
  \frametitle{Two tailed tests}
  In case of a \emph{two tailed} test, we use the $z$-score for $\frac{\alpha}{2}$:
  \[ z_{\frac{\alpha}{2}} = qnorm(1-\frac{\alpha}{2})=qnorm(0.975) \]

  We will use 2 critical values $g_1$ and $g_2$:
  \[ g_1 = \mu - z_\frac{\alpha}{2} \cdot \frac{\sigma}{\sqrt{n}} \]
  \[ g_2 = \mu + z_\frac{\alpha}{2} \cdot \frac{\sigma}{\sqrt{n}} \]

  And $H_0$ will now be rejected if $\overline{x}<g_1$ or $\overline{x}>g_2$
\end{frame}

\begin{frame}
  \frametitle{Summary}

\begin{table}
  \centering
  \begin{tabular}{l|ccc}
    \toprule
    $z$-test              & \multicolumn{3}{l}{\parbox{.7\textwidth}{Test on the population mean $\mu$ based on a sample of $n$ \textbf{independent} observations.}} \\
    \midrule
    Conditions        & \multicolumn{3}{l}{\parbox{.7\textwidth}{Population stdev $\sigma$ is known, $n>30$}} \\
    \midrule
    Test type         & Two tailed           & Left tailed & Right tailed \\
    \midrule
    $H_{0}$           & $\mu = \mu_{0}$      & $\mu = \mu_{0}$ & $\mu = \mu_{0}$  \\
    $H_{1}$           & $\mu \neq \mu_{0}$   & $\mu < \mu_{0}$ & $\mu > \mu_{0}$  \\
    Critical region   & $\left|z\right| > z_\frac{\alpha}{2}$ & $z< -z_\alpha $        & $z>z_\alpha$            \\
    Test statistic    & \multicolumn{3}{c}{$z = \frac{\overline{x} - \mu_{0}}{\frac{\sigma}{\sqrt{n}}}$} \\
    \bottomrule
  \end{tabular}
  \caption{Summary of the $z$-test}
  \label{tab:toetsingsprocedures}
\end{table}
\end{frame}

\section{Examples}
\sectionframelogo{}

\begin{frame}
  \frametitle{Example 1}
  A random sample of 50 observations has following properties: sample mean $\overline{x} = 25$ and standard deviation s = $\sqrt{55} = 7.41$
  
  We want to know if there are indications to assume that the population mean is smaller than 27.
\end{frame}

\begin{frame}
  \frametitle{Example 1}
  \begin{block}{Determine the hypotheses ($H_0$ and $H_1$)}
    $H_{0} : \mu = 27$ and $H_{1}: \mu < 27$
  \end{block}


  \begin{block}{Define the significance level}
  $\alpha = 0.05$ (and $n=50$)
  \end{block}


  \begin{block}{Calculate the test statistic}
    We will be looking at the distribution of the sample mean $M$. From the central limit theorem follows that:

    \[ M \sim Nor(\mu = 27, \sigma = \sqrt{\frac{55}{50}}) \]
    
    How extreme is $\overline{x}= 25$ in this probability distribution?
  \end{block}
\end{frame}

\begin{frame}
  \begin{block}{$p$-value}
    \[ P(M < 25) = P \left(Z < \frac{25 - 27}{\sqrt{\frac{55}{50}}} = -1.9069 \right) \approx 0.0283  \]
    
    The $p = 0.0283 < \alpha = 0.05$ which indicates we can reject  $H_{0}$
    \end{block}
  
    \begin{block}{Calculate the critical area}
    
    \[ g = \mu - z \cdot \frac{\sigma}{\sqrt{n}} \]
    \[ g = 27 - 1.645 \cdot \sqrt{\frac{55}{50}} \]
    \[ g \approx 25.2749 \]
  
    We find that $\overline{x} < g$, so we can reject $H_{0}$
  \end{block}

\end{frame}

\begin{frame}
    \centering
  \begin{tikzpicture}
    \begin{axis}[domain=24:30, samples=100, enlargelimits=false, clip=false ]
      \addplot [smooth, fill=HoGentFBO!20, domain=24:25.27] {gauss(27,1.048808848)} \closedcycle;
      \addplot [very thick,smooth,draw=HoGentFBO] {gauss(27, 1.048808848)};
      \node  at (axis cs: 25.2749, -.04){\small 25.2749};
    \end{axis}
  \end{tikzpicture}
\end{frame}

\begin{frame}
  \frametitle{Example 2}
  While doing research into the small change in our super heroes' pockets, the base assumption is that the average amount is \$25.
  
  We assume that the standard deviation is known to be $\sigma = 7$. A random sample with $n=64$ observations has a sample mean $\overline{x}$ of \$23. The significance level $\alpha = 0.05$.
\end{frame}

\begin{frame}
  \frametitle{Example 2}
  \begin{block}{Determine the hypotheses}
  $H_{0} : \mu = 25$ and $H_{1}: \mu \neq 25$ (two tailed test)
  \end{block}

  \begin{block}{Define the significance level}
  $\alpha = 0.05$ and $n=64$.
  \end{block}

  \begin{block}{Calculate the test statistic}
    $M \sim Nor(\mu = 25, \sigma = \frac{7}{\sqrt{64}})$ , $\overline{x} = 23$
  \end{block}
\end{frame}

\begin{frame}
  \begin{block}{Calculate the critical area}
    \[ \mu - z \frac{\sigma}{\sqrt{n}} = 25 - 1.959964 \cdot \frac{7}{\sqrt{64}} \approx 23.28503 \]
    
    \[ \mu + z \frac{\sigma}{\sqrt{n}} = 25 + 1.959964 \cdot \frac{7}{\sqrt{64}} \approx 26.71497 \]
    
    We find that $\overline{x}$ is inside the critical region (since $23 < 23.28$), so we can reject $H_{0}$.
  \end{block}

\end{frame}

\section{The $t$-test}
\sectionframelogo{}

\begin{frame}
  \frametitle{Assumptions $z$-test}
  
  
  \begin{itemize}
    \item The sample should be random
    \item The sample must be sufficiently large ($n > 30$)
    \item The standard deviation of the populatie, $\sigma$, is known
    \item The variation of the test statistic should be normally distributed (this is always the case if the test statistic is the sample mean, because of the central limit theorem)
  \end{itemize}

  \brightbox{If these assumptions do not hold, you \textcolor{HoGentYellow}{may not} use the $z$ test!}
\end{frame}

\begin{frame}
  \frametitle{The $t$-test}
  
  Determine critical value:
  
  \[ g = \mu \pm t_\alpha \cdot \frac{s}{\sqrt{n}} \]
  
  \begin{itemize}
    \item $t$-value is derived from Student's $t$ distribution, depends on \emph{degrees of freedom}, $n-1$
    \item Look this up in $t$-table or with R-function \texttt{qt($1-\alpha$,$n-1$)}
  \end{itemize}
  
\end{frame}

\begin{frame}
  \frametitle{Comparing two samples}
  
  Is the mean of two samples significantly different?
  
  \begin{itemize}
    \item Independent samples
    \item Paired sample
  \end{itemize}
\end{frame}

\begin{frame}
  \frametitle{Example}
  \framesubtitle{Independent samples}
  
  In a clinical trial, researchers want to determine whether a new drug has a decreased reaction time as a side effect.
  
  \begin{itemize}
    \item Control group: 6 participants receive a placebo
    \item Treatment group: 6 (other) participants receive the drug
  \end{itemize}
  
  Next, their reaction time (in ms) to a stimulus was measured:
  
  \begin{itemize}
    \item Control group: 91, 87, 99, 77, 88, 91 ~~~~~~~~~~~($\overline{x}=88.83$)
    \item Treatment group: 101, 110, 103, 93, 99, 104 ~~($\overline{y}=101.67$)
  \end{itemize}
  
  Is there a significant difference between the two groups?
\vfill
  \textbf{Remark}: There is no need to have the same number of participants in each group.
\end{frame}

\begin{frame}[fragile]
  \frametitle{Example}
  \framesubtitle{Independent samples}
\begin{enumerate}
    \item Hypothesis:
    \begin{itemize}
        \item $H_0$: $\mu_1 - \mu_2 = 0$
        \item $H_1$: $\mu_1 - \mu_2 < 0$
    \end{itemize}
    \item Significance level: $\alpha = 0.05$
    \item Test statistic:
    \begin{itemize}
        \item $\overline{x}-\overline{y} = -12.833$
        \item $\overline{x} =$ estimate for $\mu_1$ (control group) 
        \item $\overline{y} =$ estimate for $\mu_2$ (treatment group) 
    \end{itemize}
\end{enumerate}
\vfill
In R:
{\footnotesize
\begin{verbatim}
> control <-  c(91, 87, 99, 77, 88, 91)
> treatment <- c(101, 110, 103, 93, 99, 104)
> t.test(control, treatment, alternative="less",
                                      mu=0, conf.level=0.95)
\end{verbatim}
}
\end{frame}

\begin{frame}[fragile]
  \frametitle{Example}
  \framesubtitle{Independent samples}
Result in R:
{\footnotesize
\begin{Verbatim}[commandchars=\\\{\}]
	Welch Two Sample t-test

data:  control and treatment
t = -3.4456, df = 9.4797, p-value = 0.003391
alternative hypothesis: true difference in means is less than 0
\sout{95 percent confidence interval:}               {\tiny \textcolor{red}{WARNING!} This is NOT the}
\sout{-Inf    -6.044949}                    {\tiny   critical region or acceptance region}
sample estimates:
mean of x mean of y 
88.83333  101.66667   
\end{Verbatim}
}
\vfill
$\overline{x}-\overline{y}=-12.833$ is equivalent with $t=-3.4456$\\
$df=9.4797$ is calculated by \texttt{t.test()} based on $x$ and $y$\\
$p = 0.003391 < \alpha = 0.05$ so $H_1$ accepted (significant difference)
\end{frame}

\begin{frame}
  \frametitle{Example}
  \framesubtitle{Paired $t$-test}
  
  A study was performed to test whether cars get better mileage on premium gas than on regular gas. 
  
  The mileage per gallon was recorded for each of the two types of gas for ten cars.
  
  \vspace{.5cm}
  \centering
  \begin{tabular}{|l|c|c|c|c|c|c|c|c|c|c|}
  	\hline
  	Car     & 1  & 2  & 3  & 4  & 5  & 6  & 7  & 8  & 9  & 10 \\ \hline
  	Regular (y) & 16 & 20 & 21 & 22 & 23 & 22 & 27 & 25 & 27 & 28 \\ \hline
  	Premium (x) & 19 & 22 & 24 & 24 & 25 & 25 & 25 & 26 & 28 & 32 \\ \hline
  \end{tabular} 
\end{frame}

\begin{frame}[fragile]
\frametitle{Example}
\framesubtitle{Paired $t$-test}
\begin{enumerate}
    \item Hypothesis:
    \begin{itemize}
        \item $H_0$: $\overline{x-y} = 0$
        \item $H_1$: $\overline{x-y} > 0$
    \end{itemize}
    \item Significance level: $\alpha = 0.05$
    \item Test statistic:
    \begin{itemize}
        \item $\overline{x-y}$
        \item $x =$ miles per gallon with premium fuel ($\overline{x}=25.1$)
        \item $y =$ miles per gallon with regular fuel ($\overline{y}=23.1$)
    \end{itemize}
\end{enumerate}
\vfill
In R:
{\footnotesize
\begin{Verbatim}[commandchars=\\\{\}]
> regular <- c(16, 20, 21, 22, 23, 22, 27, 25, 27, 28)
> premium <- c(19, 22, 24, 24, 25, 25, 26, 26, 28, 32)
> t.test(premium, regular, alternative="greater",
                             \textcolor{red}{paired=TRUE}, mu=0, conf.level=0.95)
\end{Verbatim}
}
\end{frame}

\begin{frame}[fragile]
\frametitle{Example}
\framesubtitle{Paired $t$-test}
Resultaat in R:
{\footnotesize
\begin{Verbatim}[commandchars=\\\{\}]
Paired t-test

data:  premium and regular
t = 4.4721, df = 9, p-value = 0.0007749
alternative hypothesis: true difference in means is greater than 0
\sout{95 percent confidence interval:}                 {\tiny \textcolor{red}{WARNING!} This is NOT the}
\sout{1.180207   Inf}                        {\tiny    critical region or acceptance region}
sample estimates:
mean of the differences 
2
\end{Verbatim}
}
\vfill
$\overline{x-y}=2$ is equivalent with $t=4.4721$\\
$p = 0.0007749 < \alpha = 0.05$ so $H_1$ accepted (significant difference)
\end{frame}

\section{Errors in hypothesis tests}
\sectionframelogo{}

\begin{frame}
  \frametitle{Errors in hypothesis tests}

  \begin{table}
    \centering
    \resizebox{\textwidth}{!}{%
      \begin{tabular}{@{}l|cc@{}}
        \toprule
        & \multicolumn{2}{c}{\textbf{Actual situation}} \\
        \textbf{Conclusions}          & \textbf{$H_{0}$ is true} & \textbf{$H_{1}$ is true}     \\
        \midrule
        \textbf{$H_{0}$ accepted}& Correct decision                    & Type II error \\
        \textbf{$H_{0}$ rejected}   & Type I error          & Correct decision            \\
        \bottomrule
      \end{tabular}
    }
  \end{table}
\vfill
P(type I error) = $\alpha$ (= significance level)\\
P(type II error) = $\beta$
\vfill
It is \textbf{\textit{not}} trivial to calculate $\beta$,
but we know: if $\alpha \searrow$, then $\beta \nearrow$ 

\end{frame}
\end{document}
