%%----------------------------------------------------------------------------
%% Onderzoekstechnieken: Analyse van kwalitatieve vs kwantitatieve variabele
%%----------------------------------------------------------------------------

\documentclass[aspectratio=169]{beamer}

%==============================================================================
% Aanloop
%==============================================================================

%---------- Vormgeving --------------------------------------------------------

\usetheme{hogent}

\usecolortheme{hgwhite} % witte achtergrond, zwarte tekst

\usepackage{graphicx,multicol}
\usepackage{comment,enumerate,hyperref}
\usepackage{amsmath,amsfonts,amssymb}
\usepackage[english]{babel}
\usepackage{multirow}
\usepackage{eurosym}
\usepackage{listings}
\usepackage{textcomp}
\usepackage{framed}
\usepackage{wrapfig}
\usepackage{tabu} %needed for \tabulinesep
\usepackage{wrapfig}
\usepackage{pgf-pie}
\usepackage{pgfplots}
\usepackage{booktabs}
\usepackage{pgfplotstable}
\usepackage{changepage}
\usepackage{ulem} % for \sout{text} (strikethrough)
\usepackage{fancyvrb} % for \begin{Verbatim} (LaTeX controls within verbatim)

%---------- Configuratie ------------------------------------------------------

\pgfplotsset{compat=1.16}
\usetikzlibrary{arrows,shapes,backgrounds,positioning,shadows}
\usetikzlibrary{pgfplots.statistics}

%---------- Commando-definities -----------------------------------------------

\newcommand{\tabitem}{~~\llap{\textbullet}~~}
\newcommand{\alertbox}[2][hgblue]{%
  \setbeamercolor{alertbox}{bg=#1,fg=white}
  \begin{beamercolorbox}[sep=2pt,center]{alertbox}
    \textbf{#2}
  \end{beamercolorbox}
}
\pgfmathdeclarefunction{gauss}{2}{%
  \pgfmathparse{1/(#2*sqrt(2*pi))*exp(-((x-#1)^2)/(2*#2^2))}%
}

%---------- Info over de presentatie ------------------------------------------

\title{Chapter 6. Bivariate Analysis: qualitative - quantitative}
\subtitle{Research Techniques}
\author{Jens Buysse \and Pieter-Jan Maenhaut \and Bert {Van Vreckem}}
\date{AY 2020-2021}

%==============================================================================
% Inhoud presentatie
%==============================================================================

\begin{document}

\begin{frame}
  \maketitle
\end{frame}

\begin{frame}
  \frametitle{What's on the menu?}
  
  \tableofcontents
\end{frame}

\begin{frame}
  \frametitle{Learning Goals}
  
  \begin{itemize}
    \item Apply the $t$-test for two samples
    \item Calculate effect size using Cohen's $d$
  \end{itemize}
\end{frame}

\begin{frame}
  \frametitle{Overview}
  \centering
  \begin{tabular}{lll}
    \toprule
    \textbf{Independent} & \textbf{Dependent} & \textbf{Test/metric}        \\
    \midrule
    Qualitative             & Qualitative           & $\chi^2$-test                 \\
    &                       & Cramér's $V$                  \\
    Qualitative             & Quantitative          & two-sample $t$-test           \\
    &                       & Cohen's $d$                   \\
    Quantitative            & Quantitative          & ---                           \\
    &                       & Regression, correlation       \\
    \bottomrule
  \end{tabular}
\end{frame}

\section{Two-sample t-test}

\begin{frame}
  \frametitle{Comparing two samples}
  
  Examples of research questions:
  
  \begin{itemize}
    \item Do men earn more than women?
    \item Is it better to study without a smartphone nearby?
    \item Do ASO students achieve better results than those from TSO?
    \item How effective is a medicine?
    \item \ldots
  \end{itemize}

  In these examples, what is the independent and dependent variable?
\end{frame}

\begin{frame}
  \frametitle{Comparing two samples}
  
  Is the sample mean of two samples significantly different?
  
  \begin{itemize}
    \item Independent samples
    \item Paired samples
  \end{itemize}
\end{frame}

\begin{frame}
  \frametitle{Independent samples}
  
  In a clinical study, the aim is to determine whether a new drug has a delayed (i.e. higher) reaction speed as a side effect
  
  \begin{itemize}
    \item Control group: 6 participants receive placebo
    \item Intervention group: 6 participants receive medicine
  \end{itemize}
  
  \pause
  
  Next, the reaction speed is measured:
  
  \begin{itemize}
    \item Control group: 91, 87, 99, 77, 88, 91 ~~~~~~~~~~~($\overline{x}=88.83$)
    \item Intervention group: 101, 110, 103, 93, 99, 104 ~~($\overline{y}=101.67$)
  \end{itemize}
  
  Are there significant differences between the intervention and control group?
\end{frame}

\begin{frame}[fragile]
  \frametitle{Independent samples}
  \begin{enumerate}
    \item Hypotheses:
    \begin{itemize}
      \item $H_0$: $\mu_1 - \mu_2 = 0$
      \item $H_1$: $\mu_1 - \mu_2 < 0$
    \end{itemize}
    \item Significance level: $\alpha = 0.05$
    \item Test statistic:
    \begin{itemize}
      \item $\overline{x}-\overline{y} = -12.833$
      \item $\overline{x} =$ estimation for $\mu_1$ (control group)
      \item $\overline{y} =$ estimation for $\mu_2$ (intervention group)
    \end{itemize}
  \end{enumerate}
  \vfill
  In R:
  {\footnotesize
    \begin{Verbatim}[commandchars=\\\{\}]
    > control <-  c(91, 87, 99, 77, 88, 91)
    > intervention <- c(101, 110, 103, 93, 99, 104)
    > t.test(control, intervention, alternative="less",
    mu=0, conf.level=0.95)
    \end{Verbatim}
  }
\end{frame}

\begin{frame}[fragile]
  \frametitle{Independent samples}
  
  {\footnotesize
    \begin{Verbatim}[commandchars=\\\{\}]
    Welch Two Sample t-test
    
    data:  control and intervention
    t = -3.4456, df = 9.4797, p-value = 0.003391
    alternative hypothesis: true difference in means is less than 0
    \sout{95 percent confidence interval:}      {\tiny \textcolor{red}{REMARK!} This is unrelated to the}
    \sout{-Inf -6.044949}                       {\tiny region of acceptance or the critical region}
    sample estimates:
    mean of x mean of y 
    88.83333 101.66667
    \end{Verbatim}
  }
  \vfill
  $\overline{x}-\overline{y}=-12.833$ corresponds to $t=-3.4456$\\
  $df=9.48$ is calculated using \texttt{t.test()} based on $x$ and $y$\\
  $p = 0.003391 < \alpha = 0.05$ so reject $H_0$ (significant difference)
\end{frame}

\begin{frame}
  \frametitle{Paired samples}
  
  A study examined whether cars that run on petrol with additives have a lower consumption.
  
  For 10 cars, the consumption was measured (expressed in miles per gallon) for both fuel types:
  
  \vspace{.5cm}
  \centering
  \begin{tabular}{|l|c|c|c|c|c|c|c|c|c|c|}
    \hline
    Car           & 1  & 2  & 3  & 4  & 5  & 6  & 7  & 8  & 9  & 10 \\ \hline
    Regular petrol & 16 & 20 & 21 & 22 & 23 & 22 & 27 & 25 & 27 & 28 \\ \hline
    With additives & 19 & 22 & 24 & 24 & 25 & 25 & 25 & 26 & 28 & 32 \\ \hline
  \end{tabular} 
\end{frame}

\begin{frame}[fragile]
  \frametitle{Paired samples}
  \begin{enumerate}
    \item Hypotheses:
    \begin{itemize}
      \item $H_0$: $\overline{x-y} = 0$
      \item $H_1$: $\overline{x-y} > 0$
    \end{itemize}
    \item Significance level: $\alpha = 0.05$
    \item Test statistic:
    \begin{itemize}
      \item $\overline{x-y}$
      \item $x =$ miles per gallon with additives ($\overline{x}=25.1$)
      \item $y =$ miles per gallon regular petrol ($\overline{y}=23.1$)
    \end{itemize}
  \end{enumerate}
  \vfill
  In R:
  {\footnotesize
    \begin{Verbatim}[commandchars=\\\{\}]
    > regular    <- c(16, 20, 21, 22, 23, 22, 27, 25, 27, 28)
    > additives  <-c(19, 22, 24, 24, 25, 25, 26, 26, 28, 32)
    > t.test(additives, regular, alternative="greater",
    \textcolor{red}{paired=TRUE}, mu=0, conf.level=0.95)
    \end{Verbatim}
  }
\end{frame}

\begin{frame}[fragile]
  \frametitle{Paired samples}
  
  {\footnotesize
    \begin{Verbatim}[commandchars=\\\{\}]
    Paired t-test
    
    data:  additives and regular
    t = 4.4721, df = 9, p-value = 0.0007749
    alternative hypothesis: true difference in means is greater than 0
    \sout{95 percent confidence interval:}      {\tiny \textcolor{red}{REMARK!} This is unrelated to the}
    \sout{1.180207      Inf}                    {\tiny region of acceptance or the critical region}
    sample estimates:
    mean of the differences 
    2
    \end{Verbatim}
  }
  \vfill
  $\overline{x}-\overline{y}=2$ corresponds to $t=4.4721$\\
  $p = 0.0007749 < \alpha = 0.05$ so reject $H_0$ (significant difference)
\end{frame}

\section{Effect size}

\begin{frame}
  \frametitle{Effect size}
  
  \alertbox{The \textcolor{hgyellow}{effect size} is a metric which expresses how great the difference between two groups is}
  
  \begin{itemize}
    \item Control group vs. intervention group
    \item Can be used in addition to hypothesis test
    \item Often used in educational sciences
    \item There are several definitions, here: \textit{Cohen's $d$}
  \end{itemize}
\end{frame}

\begin{frame}
  \frametitle{Cohen's $d$}
  
  \[ d = \frac{\overline{x_1} - \overline{x_2}}{s} \]
  
  with $\overline{x_1}$, $\overline{x_2}$ the sample means
  
  and $s$ a standard deviation of both groups combined:
  
  \[ s = \sqrt{\frac{(n_1 - 1) s_1^2 + (n_2 - 1) s_2^2}{n_1 + n_2 -2}} \]
  
  with $n_1$, $n_2$ the sample sizes, $s_1$, $s_2$ the standard deviation of both groups
  
\end{frame}

\begin{frame}
  \frametitle{Interpretation Cohen's $d$}
  
  \begin{columns}
    \column{.5\textwidth}
    
    \begin{center}
      \begin{tabular}{ll}
        \toprule
        $|d|$  & \textbf{Effect Size} \\
        \midrule
        0.01 & Very small             \\
        0.2  & Small                  \\
        0.5  & Average                \\
        0.8  & Large                  \\
        1.2  & Very large             \\
        2.0  & Huge                   \\
        \bottomrule
      \end{tabular}
    \end{center}
    
    \column{.5\textwidth}
    
    In educational sciences (John Hattie):
    
    \begin{itemize}
      \item 0,4 = tipping point for desired effects
      \item effect size $d = 1$: process material from 1y in 6m!
    \end{itemize}
    
    Cfr. for example \url{http://www.evidencebasedteaching.org.au/hatties-2017-updated-list/}
  \end{columns}
\end{frame}

\begin{frame}
  \frametitle{Typical approach research in education}
  
  \begin{itemize}
    \item Research question: Is X a good learning strategy, in other words, does this have a positive effect on final results?
    \item Control group uses ``normal'', ``classic'' technique
    \item X is used in the intervention group
    \item Followed by an evaluation moment
    \item Determine scores, calculate $d$
  \end{itemize}
\end{frame}

\end{document}