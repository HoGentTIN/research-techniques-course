%%----------------------------------------------------------------------------
%% Onderzoekstechnieken: Toetsingsprocedures
%%----------------------------------------------------------------------------

\documentclass[aspectratio=169]{beamer}

%==============================================================================
% Aanloop
%==============================================================================

%---------- Vormgeving --------------------------------------------------------

\usetheme{hogent}

\usecolortheme{hgwhite} % witte achtergrond, zwarte tekst

\usepackage{graphicx,multicol}
\usepackage{comment,enumerate,hyperref}
\usepackage{amsmath,amsfonts,amssymb}
\usepackage[dutch]{babel}
\usepackage{multirow}
\usepackage{eurosym}
\usepackage{listings}
\usepackage{textcomp}
\usepackage{framed}
\usepackage{wrapfig}
\usepackage{tabu} %needed for \tabulinesep
\usepackage{wrapfig}
\usepackage{pgf-pie}
\usepackage{pgfplots}
\usepackage{booktabs}
\usepackage{pgfplotstable}
\usepackage{changepage}
\usepackage{ulem} % for \sout{text} (strikethrough)
\usepackage{fancyvrb} % for \begin{Verbatim} (LaTeX controls within verbatim)

%---------- Configuratie ------------------------------------------------------

\pgfplotsset{compat=1.16}
\usetikzlibrary{arrows,shapes,backgrounds,positioning,shadows}
\usetikzlibrary{pgfplots.statistics}

%---------- Commando-definities -----------------------------------------------

\newcommand{\tabitem}{~~\llap{\textbullet}~~}
\newcommand{\alertbox}[2][hgblue]{%
  \setbeamercolor{alertbox}{bg=#1,fg=white}
  \begin{beamercolorbox}[sep=2pt,center]{alertbox}
    \textbf{#2}
  \end{beamercolorbox}
}
\pgfmathdeclarefunction{gauss}{2}{%
  \pgfmathparse{1/(#2*sqrt(2*pi))*exp(-((x-#1)^2)/(2*#2^2))}%
}

%---------- Info over de presentatie ------------------------------------------

\title{Chapter 5. Testing Procedures}
\subtitle{Research techniques}
\author{Jens Buysse \and Wim {De Bruyn} \and Pieter-Jan Maenhaut \and Bert {Van Vreckem}}
\date{AY 2019-2020}

%==============================================================================
% Inhoud presentatie
%==============================================================================

\begin{document}

\begin{frame}
  \maketitle
\end{frame}

\begin{frame}
  \frametitle{What's on the menu?}
  
  \tableofcontents
\end{frame}

\begin{frame}
  \frametitle{Learning Goals}
  
  \begin{itemize}
    \item Statistical tests: concepts
    \item Procedure for statistical tests
    \item Apply the $z$- and $t$-test
  \end{itemize}
\end{frame}


\section{Hypothesis Tests}

\begin{frame}
  \frametitle{Statistical test for a hypothesis}
  
  \begin{description}
    \item[Hypothesis] Idea that has yet to be proven: statement regarding the numeric value of a population parameter
    \item[Hypothesis Test] verification of a statement about the values of one or multiple population parameters
    \item[Null Hypothesis ($H_0$)]  We will try to disprove this hypothesis using proof by contradiction
    \item[Alternative Hypothesis ($H_1$, $H_a$)] We want to prove this hypothesis
  \end{description}
\end{frame}

\begin{frame}
  \frametitle{Elements of a testing procedure}
  
  \begin{description}
    \item[Test Statistic] The variable that is calculated from the sample
    \item[Region of Acceptance] The region of values \alert{supporting} the null hypothesis
    \item[Critical Region / Region of Rejection] The region of values \alert{rejecting} the null hypothesis
    \item[Significance Level] The probability of rejecting a true null hypothesis $H_0$
  \end{description}
\end{frame}

\section{Method}

\begin{frame}
  \frametitle{Method}
  
  \begin{enumerate}
    \item Formulate both hypotheses ($H_0$ en $H_1$)
    \item Determine the significance level ($\alpha$)
    \item Calculate the test statistic
    \item Determine the critical region or the probability value
    \item Draw conclusions
  \end{enumerate}
\end{frame}

\begin{frame}[plain]
  \frametitle{Hypotheses about superheroes}
  \centering
  \includegraphics[height=.85\textheight]{les5-heroes}
\end{frame}

\begin{frame}
  \frametitle{A superhero rescues 3.3 persons a day}
  
  \begin{columns}
    \column{.5\textwidth}
    \centering
    \includegraphics[width=3cm]{les5-gered1}
    
    \includegraphics[width=3cm]{les5-gered2}
    \column{.5\textwidth}
    \centering
    \includegraphics[width=3cm]{les5-gered3}
    
    \includegraphics[width=3cm]{les5-gered4}
  \end{columns}
  
  \vfill
  \centering
  \small{Source: \url{http://www.cracked.com/quick-fixes/4-people-who-saved-day-while-dressed-as-superheroes/}}
\end{frame}

\begin{frame}
  Assume that, over a period of 30 days, $3.483$ people were saved a day on average ($\overline{x}=3.483$, $n=30$)
  \vfill
  \begin{enumerate}
    \item Hypothesis: $H_0: \mu = 3.3$; $H_1: \mu > 3.3$
    \item Significance level: $\alpha = 0.05$
    \item Test statistic: $\overline{x} = 3.483$
  \end{enumerate}
  \vfill
  Standard deviation of the population (assumed to be known): $\sigma = 0.55$
\end{frame}

\begin{frame}
  \frametitle{Calculate test statistic}
  
  \bigskip
  
  Based on the central limit theorem, we know that: $M \sim Nor(\mu = 3.3; \sigma = \frac{0.55}{\sqrt{30}}=0.1)$
  
  \centering
  \begin{tikzpicture}[scale=.8]
  \begin{axis}[domain=3.0:3.6, ymax=4.2, samples=100, enlargelimits=false ]
  \addplot [very thick,smooth,draw=hgblue] {gauss(3.3, 0.10041580220928045413)};
  \node [coordinate, pin={$\overline{x}= 3.483$}] at (axis cs: 3.483, 0){};
  \end{axis}
  \end{tikzpicture}
\end{frame}

\section{Probability Value}

\begin{frame}
  \frametitle{Probability Value}
  
  \alertbox{The \textcolor{hgyellow}{$p$-value} is the probability, if the null hypothesis is true, to obtain a value for the test statistic that is at least as extreme as the observed value.}
  
  \begin{itemize}
    \item $p$-value $< \alpha \Rightarrow$ reject $H_{0}$: the discovered value of $\overline{x}$ is too extreme;
    \item $p$-value $\geq \alpha \Rightarrow$ do not reject $H_{0}$: the discoverd value of $\overline{x}$ can still be explained by coincidence.
  \end{itemize}
\end{frame}

\begin{frame}
  \frametitle{Probability Value}
  
  \bigskip
  
  \centering
  \begin{tikzpicture}[scale=0.6]
  \begin{axis}[domain=-3.5:3.5, ymax=0.42, samples=100, enlargelimits=false, clip=false ]
  \addplot [smooth, fill=black!20, domain=1.822:3.5] {gauss(0,1)} \closedcycle;
  \addplot [very thick,smooth,draw=hgblue] {gauss(0, 1)};
  \node  at (axis cs: 1.822, -.04){\small 1.822};
  \draw [-latex](axis cs:2.1,0.02) -- (axis cs:2.35,0.06);
  \node [text width=0.5cm] at (axis cs: 2.5, .095){$p= 0.0344$};
  \end{axis}
  \end{tikzpicture}
  
  \[ P(M > 3.483) = P \left(Z> \frac{3.483 - 3.3}{\frac{\sigma}{\sqrt{n}}}\right) = P (Z > 1.822) = 0.0344 \]
\end{frame}

\section{Critical Region}

\begin{frame}
  \frametitle{Critical Region}
  
  \bigskip
  
  \alertbox{The \textcolor{hgyellow}{critical region} is the collection of all values of the test statistic for which we can reject the null hypothesis.}
  
  We look for a critical value $g$ for which:
  \[ P(M > g) = \alpha \]
  
  Determine $z_\alpha$ for which:
  \[ P(Z > z_\alpha) = \alpha \Rightarrow g = \mu + z_\alpha . \frac{\sigma}{\sqrt{n}} \]
  
  \begin{itemize}
    \item Left of $g$: region of acceptance (do not reject $H_0$)
    \item Right of $g$: critical region (reject $H_0$)
  \end{itemize}
  
\end{frame}


\begin{frame}
  \frametitle{Critical Region}
  
  \centering
  \begin{tikzpicture}[scale=.9]
  \begin{axis}[domain=-3.5:3.5, ymax=0.42, samples=100, enlargelimits=false, clip=false ]
  \draw [-](axis cs:1.645,-0.03) -- (axis cs:1.645,0.06);
  \addplot [smooth, fill=black!20, domain=1.645:3.5] {gauss(0,1)} \closedcycle;
  \addplot [very thick,smooth,draw=hgblue] {gauss(0, 1)};
  \node [text width=0.8cm] at (axis cs: 1.645, -.06){$z_{\alpha}= 1.645$};
  \draw [-latex](axis cs:2.1,0.02) -- (axis cs:2.35,0.06);
  \node [text width=0.5cm] at (axis cs: 2.5, .095){$\alpha= 0.05$};
  \end{axis}
  \end{tikzpicture}
  
  significance level $\alpha = 0.05$ $\Rightarrow$ critical value $z_{\alpha}$=1.645\\
  ($z_{\alpha} = $\texttt{qnorm(0.95)})
  
\end{frame}

\begin{frame}
  \frametitle{Left-tailed testing}
  
  What would you have to change in the equation in order to calculate the correct critical value?
  \pause
  Answer:
  \[g = \mu \alert{-} z \times \frac{\sigma}{\sqrt{n}} \]
  because
  \[ P(M < g) = P\left(Z < \frac{g - \mu}{\frac{\sigma}{\sqrt{n}}}\right) = 0.05 \]
  
\end{frame}

\begin{frame}
  \frametitle{Left-tailed testing}
  Because of symmetry: 
  \[ P\left(Z > - \left( \frac{g - \mu}{\frac{\sigma}{\sqrt{n}}} \right) \right) = 0.05 \]
  
  The corresponding $z$-value is 1.645, and therefore:
  
  \begin{align*}
  z = \frac{-g + \mu}{\frac{\sigma}{\sqrt{n}}} &\Leftrightarrow -g = \frac{\sigma}{\sqrt{n}} z_\alpha - \mu \\
  & \Leftrightarrow g = \mu - z_\alpha \frac{\sigma}{\sqrt{n}}
  \end{align*}
  
\end{frame}

\begin{frame}
  \frametitle{Two-tailed testing}
  Sometimes it can be necessary to perform a two-tailed test. In this case, two critical values need to be calculated, namely the left and right critical value.
  
  \begin{equation}
  g = \mu \pm z_{\alpha/2} \times \frac{\sigma}{\sqrt{n}}
  \label{eq:critical-values}
  \end{equation}
\end{frame}

\begin{frame}[plain]
  \frametitle{Summary}
  
  \begin{table}
    \centering
    \begin{tabular}{l|ccc}
      \toprule
      Goal              & \multicolumn{3}{l}{\parbox{.7\textwidth}{Test regarding the value of the population mean $\mu$ using a sample of $n$ independent values}} \\
      \midrule
      Prerequisite      & \multicolumn{3}{l}{\parbox{.7\textwidth}{De population has a random distribution, $n$ is sufficiently large}} \\
      \midrule
      Test Type         & Two-tailed           & Left-tailed & Right-tailed \\
      \midrule
      $H_{0}$           & $\mu = \mu_{0}$      & $\mu = \mu_{0}$ & $\mu = \mu_{0}$  \\
      $H_{1}$           & $\mu \neq \mu_{0}$   & $\mu < \mu_{0}$ & $\mu > \mu_{0}$  \\
      Critical Region   & $\left|\overline{x}\right| > g$ & $\overline{x}< -g $        & $\overline{x}>g$            \\
      Test statistic    & \multicolumn{3}{c}{$z = \frac{\overline{x} - \mu_{0}}{\frac{\sigma}{\sqrt{n}}}$} \\
      \bottomrule
    \end{tabular}
    \caption{Summary of Testing Procedures}
    \label{tab:testing-procedures}
  \end{table}
\end{frame}

\section{Examples}

\begin{frame}
  \frametitle{Example 1}
  When drawing a random sample consisting of 50 observations from a population with variance $\sigma^2 = 55$ we obtain as sample mean $\overline{x} = 25$.
  
  We now want to find out if there is a reason to assume that the population mean is smaller than 27.
  
\end{frame}

\begin{frame}
  \frametitle{Example 1}
  
  \begin{description}
    \item[1] Formulate both hypotheses
    
    $H_{0} : \mu = 27$ en $H_{1}: \mu < 27$.
    
    \item[2] Determine significance level $\alpha$ and sample size $n$
    
    $\alpha = 0.05$ en $n=50$
    
    \item[3] Test statistic \& value: sample mean $\overline{x} = 25$
    
    
  \end{description}
\end{frame}

\begin{frame}
  \frametitle{Example 1 (cont.)}
  
  \begin{description}
    \item[4a] Probability Value
    
    According to the central limit theorem:
    
    \[ M \sim Nor(\mu = 27, \frac{\sigma}{\sqrt{n}}) \]
    \[ z = \frac{\overline{x} - \mu}{\frac{\sigma}{\sqrt{n}}} = \frac{25-27}{\sqrt\frac{55}{50}} \approx -1.91\]
    
    We therefore have a probability value of $0.0281$.
    
    Using a significance level of 0.05, we can reject $H_{0}$.
  \end{description}
  
\end{frame}

\begin{frame}
  \frametitle{Example 1 (cont.)}
  
  \begin{description}
    
    \item[4b] Calculate and plot the critical region
    
    \begin{align*}
    g &= \mu - z \times \frac{\sigma}{\sqrt{n}} \\
    &= 27 - 1.645 \times \sqrt{\frac{\sigma}{n}} \\
    &= 25.27470944
    \end{align*}
    
    We have that $\overline{x} < g$ and therefore we can reject $H_{0}$.
  \end{description}
\end{frame}

\begin{frame}
  \frametitle{Example 1 (cont.)}
  
  \bigskip
  \centering
  \begin{tikzpicture}
  \begin{axis}[domain=24:30, samples=100, enlargelimits=false, clip=false ]
  \addplot [smooth, fill=cyan!20, domain=24:25.27] {gauss(27,1.048808848)} \closedcycle;
  \addplot [very thick,smooth,draw=hgblue] {gauss(27, 1.048808848)};
  \node  at (axis cs: 25.27, -.04){\small 25.27};
  \end{axis}
  \end{tikzpicture}
\end{frame}

\begin{frame}
  \frametitle{Example 1 (cont.)}
  
  \begin{description}
    
    \item[5] Conclusion
    
    We can conclude, based on the sample, that $\mu < 27$ for a significance level of $0.05$
  \end{description}
\end{frame}


\begin{frame}
  \frametitle{Example 2}
  In a reseach about the amount of change in the pockets of our superheroes, researchers state that on average a superhero carries \euro{25} of cash.
  We assume that the standard deviation of the population $\sigma = 7$.
  For a random sample of size $n=64$, the average amount of money a superhero carries is $\overline{x} = $ \euro{23}. 
  For the significance level, $\alpha = 0.05$ is selected.
\end{frame}

\begin{frame}
  \frametitle{Example 2}
  
  \begin{description}
    \item[1] Formulate both hypotheses
    
    $H_{0} : \mu = 25$ en $H_{1}: \mu \neq 25$
    
    \item[2] Determine significance level $\alpha$ and sample size $n$
    
    $\alpha = 0.05$ en $n=64$.
    
    \item[3] Test statistic \& value: $\overline{x} = 23$
  \end{description}
  
\end{frame}

\begin{frame}
  \frametitle{Example 2 (cont.)}
  
  \begin{description}
    \item[4b] Calculate the critical region
    
    \[ g_{1} = \mu - z \times \frac{\sigma}{\sqrt{n}} = 23.28 \]
    \[ g_{2} = \mu + z \times \frac{\sigma}{\sqrt{n}} = 26.72 \]
    
    We have that $\overline{x}$ is inside the critical region (because $23 < 23.28$) so we can reject $H_{0}$.
    \item[5] Based on this sample we can conclude that superheroes carry on average \textit{less} than \EUR{25}, using a significance level of 5\%
  \end{description}
\end{frame}

\section{The $t$-test}

\begin{frame}
  \frametitle{Requirements for $z$-test}
  
  
  \begin{itemize}
    \item The sample needs to be random
    \item The sample size needs to be sufficiently large ($n \ge 30$)
    \item The test statistic needs to have a normal distribution
    \item The standard deviation of the population, $\sigma$, is known
  \end{itemize}
  
  Sometimes these assumptions will not hold and in this case we can \emph{not} use the $Z$-test!
\end{frame}

\begin{frame}
  \frametitle{The $t$-test}
  
  Determine critical value:
  
  \[ g = \mu \pm t \times \frac{s}{\sqrt{n}} \]
  
  \begin{itemize}
    \item $t$-value is derived from the Student's $t$-distribution, based on the number of \textit{degrees of freedom}, $n-1$
    \item Look for value using $t$-table or using the function \texttt{pt} in R
    \item Apart from this, the procedure is identical to the procedure of the $z$-test
  \end{itemize}
  
\end{frame}

\section{Errors in Hypothesis Tests}

\begin{frame}
  \frametitle{Errors in Hypothesis Tests}
  
  \begin{table}
    \centering
    \resizebox{\textwidth}{!}{%
      \begin{tabular}{@{}l|cc@{}}
        \toprule
        & \multicolumn{2}{c}{\textbf{Reality}} \\
        \textbf{Conclusion}          & \textbf{$H_{0}$ True} & \textbf{$H_{1}$ True}     \\
        \midrule
        \textbf{$H_{0}$ not rejected}& Correct                       & Type II error (false negative) \\
        \textbf{$H_{0}$ rejected}    & Type I error (false positive) & Correct                        \\
        \bottomrule
      \end{tabular}
    }
    \label{tab:hypfouten}
  \end{table}
  
  P(type I error) = $\alpha$ (= significance level)
  
  P(type II error) = $\beta$
  
  Calculating $\beta$ is \textbf{\textit{not}} trivial, but if $\alpha \searrow$ then $\beta \nearrow$ 
\end{frame}

\end{document}