\chapter{Getting started}
\label{ch:getting-started}

\section{Software installation}
\label{sec:installatie-software}

For this course you will need the following software:

\begin{itemize}
  \item Git client;
  \item \LaTeX{} compiler;
  \item \LaTeX{} editor;
  \item Jabref;
  \item R;
  \item Rstudio (IDE voor R).
\end{itemize}

Instructions to install this software can be found on:
\begin{itemize}
    \item Chamilo, under ``Documents'', ResearchTechniques-Introduction.pdf
    \item GitHub (https://github.com/HoGentTIN/research-techniques-course), check README.md
\end{itemize}

\section{Usage of R}

R is software for data manipulation, calculation and graphic representation.

In R, to get help about a function or anything else: just type it with a question mark in front of it.
\begin{lstlisting}
> ?solve
\end{lstlisting}

\subsection{R environment and workspace}

The R command
\begin{lstlisting}
> objects()
\end{lstlisting}
returns a list of all objects (variables, arrays, ...) that have been created in the workspace up to that point.
To remove objects from the workspace use \texttt{rm}:
\begin{lstlisting}
> rm (x, y, z, inkt, junk, temp, foo, bar)
\end{lstlisting}

\subsection{Creating a ``collection''}

Create a new variable ``x'', and assign a collection of 5 numbers to it:
\begin{lstlisting}
> x <- c(10.4, 5.6, 3.1, 6.4, 21.7)
\end{lstlisting}
You can also call this an array, a vector or a list.

To view the content of a variable, just type it an press ENTER:
\begin{lstlisting}
> x
[1] 10.4  5.6  3.1  6.4 21.7
\end{lstlisting}

You can also select just 1 number:
\begin{lstlisting}
> x[2]
[1] 5.6
\end{lstlisting}

\subsection{Reading a csv file}

In a lot of cases, data is available in a csv file (csv = comma separated values). In a csv file, each line contains a row of values that can be numbers or letters, and each value is separated by a comma. In general, the first row contains a list of labels. The idea is that the labels in the top row are used to refer to the different variables per row.

The command to read the csv file is \texttt{read.csv}.

\begin{exercise}
    Use \texttt{?read.csv} to check what the parameters of the command are. Then try reading the file \texttt{research-techniques-course/syllabus/data/computers.csv}.
\end{exercise}

With \texttt{dir()}, \texttt{getwd()} and \texttt{setwd(``DIRNAME'')} you should be able to navigate to the correct directory. And read the file \texttt{computers.csv} in the variable \texttt{mydata}.

The csv file comes from the publication of \autocite{Stengos2005}. This dataset contains data from 1993 to 1995 on computer prices. You can check the effect of the addition of CD-ROM drive on the price of the computer or the effect of clock speed on the price.

\begin{lstlisting}
> getwd()
[1] "C:/Users/wgoe370/git/research-techniques-course"
> dir()
[1] "case-research-process"  "codingexamples"  "compile-all-tex.sh"  "exercises"  "README.md"  "slides"
> setwd("syllabus/data/")
> getwd()
[1] "C:/Users/wgoe370/git/research-techniques-course/syllabus/data"
> mydataset = read.csv("computers.csv")
\end{lstlisting}

To check which columns are defined, use \texttt{names()}:
\begin{lstlisting}
> names(mydataset)
[1] "price"   "speed"   "hd"      "ram"     "screen"  "cd"      "multi"   "premium" "ads"     "trend"
\end{lstlisting}

\subsection{Attributes}

Wanneer u het commando \texttt{read.csv} gebruikt, gebruikt R een specifiek soort variabele, dat een dataframe heet. Alle gegevens worden opgeslagen in het dataframe als afzonderlijke kolommen. Als u niet zeker weet wat voor variabele u hebt dan kunt u de opdracht \texttt{attributes} gebruiken. Hiermee worden alle dingen vermeld die R gebruikt om de variabele te beschrijven:

\begin{lstlisting}
attributes(mydataset)
$names
[1] "price"   "speed"   "hd"      "ram"     "screen"  "cd"      "multi"   "premium" "ads"     "trend"  

$class
[1] "tbl_df"     "tbl"        "data.frame"

$row.names
[1]    1    2    3    4    5    6    7    8    9   10   11   12   13   14   15   16   17   18   19   20   21   22   23   24   25   26   27
[28]   28   29   30   31   32   33   34   35   36   37   38   39   40   41   42   43   44   45   46   47   48   49   50   51   52   53   54
... snip ...
[973]  973  974  975  976  977  978  979  980  981  982  983  984  985  986  987  988  989  990  991  992  993  994  995  996  997  998  999
[1000] 1000
[ reached getOption("max.print") -- omitted 5259 entries ]

$spec
cols(
price = col_integer(),
speed = col_integer(),
hd = col_integer(),
ram = col_integer(),
screen = col_integer(),
cd = col_character(),
multi = col_character(),
premium = col_character(),
ads = col_integer(),
trend = col_integer()
)
\end{lstlisting}

\subsection{Data types}

We kijken naar enkele manieren waarop R gegevens kan opslaan en organiseren. Dit is echter een inleiding dus beschouwen we maar een kleine subset van de verschillende datatypes die door R worden herkend. 

\subsubsection{Numbers}

De meest eenvoudige manier om een nummer op te slaan is om een variabele van een enkel getal te nemen:

\begin{lstlisting}
> a <- 3
>
\end{lstlisting}

Hiermee kunt u allerlei basisoperaties doen en opslaan:

\begin{lstlisting}
> b <- sqrt(a*a+3)
> b
[1] 3.464102
\end{lstlisting}

Als u een lijst met nummers wilt initialiseren, kan het \texttt{numeric} commando worden gebruikt. Om bijvoorbeeld een lijst van 10 nummers te maken, gebruikt u de volgende opdracht. Je kan ook kijken naar het type van de variabele.

\begin{lstlisting}
> a <- numeric(10)
> a
[1] 0 0 0 0 0 0 0 0 0 0
> typeof(a)
[1] "double"
\end{lstlisting}

\subsubsection{Strings}

Een tekenreeks wordt gespecificeerd door gebruik te maken van aanhalingstekens. Zowel enkelvoudige als dubbele aanhalingstekens zullen werken:

\begin{lstlisting}
> a <- "hello"
> a
[1] "hello"
> b <- c("hello","there")
> b
[1] "hello" "there"
> b[1]
[1] "hello"
\end{lstlisting}

\subsubsection{Factors}

Vaak bevat een experiment proeven voor verschillende niveaus van een bepaalde variabele. Bijvoorbeeld een nominale variabele die gecodeerd wordt met een integer. De verschillende niveaus worden ook factoren genoemd.

Je geeft aan dat een variabele een factor is met behulp van het \texttt{factor} commando. 

\subsubsection{Data frames}

Data kan worden opgeslaan aan de hand van dat frames. Dit is een manier om verschillende vectoren van verschillende types te nemen en ze op te slaan in dezelfde variabele. De vectoren kunnen van alle soorten zijn. Een dataframe kan bijvoorbeeld verschillende vectoren bevatten, en elke lijst kan een vector zijn van factoren, strings of nummers.

Er zijn verschillende manieren om gegevensframes te maken en te manipuleren. De meeste zijn buiten het bereik van deze introductie. Ze worden hier alleen genoemd om een meer volledige beschrijving te geven. 

\lstinputlisting{data/dataframe.R}

\subsubsection{Logische variabelen}

Een ander belangrijk gegevenstype is het logische type. Er zijn twee vooraf gedefinieerde variabelen, \texttt{TRUE} en \texttt{FALSE}.

\subsubsection{Tables}

Een andere  manier om informatie op te slaan is in een tabel.  We kijken alleen maar naar het maken en defini\"eren van tabellen. 

\lstinputlisting{data/tables.R}
Als je rijen wilt toevoegen aan uw tabel, voeg dan nog een vector toe als argument van de tabelopdracht. In het onderstaande voorbeeld hebben wij twee vragen. In de eerste vraag staan de reacties  'Nooit', 'Soms' of 'Altijd'. In de tweede vraag staan de reacties 'Ja', 'No' of 'Maybe'. De set van vectoren 'a,' En "b" bevatten het antwoord voor elke meting. Het derde punt in "a" is hoe de derde persoon op de eerste vraag reageerde en het derde punt in "b" is hoe de derde persoon op de tweede vraag reageerde.

\lstinputlisting{data/twotables.R}

\subsubsection{Matrix}

Een matrix is een verzameling van gegevens die zijn aangebracht in een tweedimensionale rechthoekige indeling. Een voorbeeld van een matrix is bijvoorbeeld als volgt:

\[
\begin{bmatrix}
2 & 3 \\ 
4 & 5  
\end{bmatrix}
\]

\lstinputlisting{data/matrix.R}

\section{Oefeningen}

\begin{exercise}
  Bekijk de dataset mtcars. Geef de waarde terug voor de eerste rij, tweede kolom. Geef ook het aantal rijen, het aantal kolommen. Geef ook een preview van het volledige data frame. Geef enkel de kolom terug met de definities van de cylinders. Om een data frame te bekomen met de twee kolommen mpg en hp, pakken we de kolomnamen in een indexvector in met single square bracket operator. Probeer ook eens op te zoeken hoe je een rijrecord van de ingebouwde data set mtcars bepaalt.
\end{exercise}

\begin{exercise}
  Maak zelf een willekeurige datafile aan in excel en probeer deze in te lezen in R. Zijn er nog dataformaten die ondersteund worden door R?
\end{exercise}

\begin{exercise}
  Genereer een $4x5$ array en noem die $x$. Geneer daarna een $3x2$ array waar de eerste kolom de rijindex kan zijn van $x$ en de tweede kolom een kolomindex voor $x$. Vervang de elementen gedefinieerd door de index in $i$ in $x$ door 0. 
\end{exercise}

\begin{exercise}
  Genereer een vector waar een voornaam en een achternaam in komen. Benoem ook de naam van de kolommen. Geef daarna ook voornaam terug van het eerste element van de array. 
\end{exercise}

\begin{exercise}
  Probeer voor de datafile \texttt{rainforest} in de library \texttt{DAAG} te tellen hoeveel rijen er zijn per species die volledig en compleet zijn (dus geen n.a. bevatten). Je kan hiervoor \texttt{with, table, complete.cases} voor gebruiken. 
\end{exercise}
