%==============================================================================
% Template for the assignment Bachelor thesis proposal
%==============================================================================
% Based on the LaTeX-template ‘Stylish Article’ (zie artikeltin.cls)
% Authors: Jens Buysse, Bert Van Vreckem
%
% TODO: First of all, change the name of this file to your own name.
%
% Compiling this document:
% 1) latexmk -pdf <FILENAME>
% 2) biber <FILENAME>
% 3) latexmk -pdf <FILENAME>

\documentclass[fleqn,10pt]{artikeltin}

%------------------------------------------------------------------------------
% Metadata
%------------------------------------------------------------------------------

\JournalInfo{UCG Business and Information Management} % Journal information
\Archive{Research Techniques 2016 - 2017} % Additional notes (e.g. copyright, DOI, review/research article)

%---------- Title & author ----------------------------------------------------

% TODO: Enter a working title for your own proposal here
\PaperTitle{Proposal title}
\PaperType{Bachelor thesis proposal} % Type document

% TODO: Replace with your own name as author, provide your email address. If you already have a co-promotor, you can add them as co-author.
\Authors{Jens Buysse\textsuperscript{1}, Anita Bernard\textsuperscript{2}, Bert Van Vreckem\textsuperscript{3}} % Authors
\affiliation{\textbf{Contact:}
  \textsuperscript{1} \href{mailto:jens.buysse@hogent.be}{jens.buysse@hogent.be};
  \textsuperscript{2} \href{mailto:anita.bernard@hogent.be}{anita.bernard@hogent.be};
  \textsuperscript{3} \href{mailto:bert.vanvreckem@hogent.be}{bert.vanvreckem@hogent.be}}

%---------- Abstract ----------------------------------------------------------

\Abstract{ Write a summary of your proposal here, as a continuous text of one paragraph. \emph{This is the last thing to write!} Finish your proposal first. Elements that should be present in the abstract: \textbf{Context} (why is this subject important?), \textbf{Need} (Why does this have to be researched?), \textbf{Task} (What will you do exactly?), \textbf{Object} (What are the contents of this document?), \textbf{Result} (What do you expect the result to be?), \textbf{Conclusion} (What conclusions do you expect to draw? How are these relevant for the audience?), \textbf{Perspective} (Are there possibilities for future work?). }

%---------- Research domain and keywords --------------------------------------
% TODO: Define the research domain and provade a few keywords

% The first keyword describes the research domain. Choose from the list below
% - Mobile application development
% - Web application development
% - Application development (other types, specify)
% - System and network administration
% - Mainframe
% - E-business
% - Databases and big data
% - Machine learning and artificial intelligence
% - Others (please specify)

\newcommand{\keywordname}{Keywords} % Defines the keywords heading name
\Keywords{Research domain --- keyword1 --- keyword2 --- keyword3} % Keywords

%---------- Title, table of contents ------------------------------------------
\begin{document}

%\flushbottom % Makes all text pages the same height
\maketitle % Print the title and abstract box
\tableofcontents % Print the contents section
\thispagestyle{empty} % Removes page numbering from the first page

%------------------------------------------------------------------------------
% Body
%------------------------------------------------------------------------------

\section{Introduction} % The \section*{} command stops section numbering
\label{sec:introduction}

Introduce the subject, research objectives, research question(s).

\section{State-of-the-art}
\label{sec:state-of-the-art}

Literature review goes here.

% For citing sources from literature, there are two important commands:
%
% \autocite{KEY} => (Author, year)
%
%   Use this if the name of the author is NOT grammatically a part of the sentence.
%
% Example, definition of a technical term:
%
%   \emph{Serverless architectures} refer to applications that significantly
%   depend on third-party services or on custom code that's run in ephemeral
%   containers~\autocite{Roberts2016}.
%
%   =>
%
%  [...] ephemeral containers (Roberts, 2016).
%
% \textcite{KEY} => Author (year)
%
%   Use this if the author's name is part of the sentence.
%
% Example:
%
%   \textcite{VanHulle1998} developed a variant of Kohonen Self-Organizing Maps that
%   converge much more quickly.
%
%  =>
% 
%  Van Hulle (1998) developed a variant [...]

\section{Methodology}
\label{sec:methodology}

How are you going to conduct the research? Which research methods are you going to apply to respond to each of your research questions?

\section{Expected results}
\label{sec:expected_results}

Describe what results you expect.

\section{Expected conclusions}
\label{sec:expected_conclusions}

Describe what you expect to come out of this research and how this will affect the intended audience. Do you already see possibilities for further research?

%------------------------------------------------------------------------------
% Bibliography
%------------------------------------------------------------------------------
% TODO: Add all sources to the BibTeX-file ``biblio.bib''
% Use JabRef to edit this file and don't forget to enable compatibility with 
% Biber/BibLaTeX (File > Switch to BibLaTeX mode)

\phantomsection
\printbibliography[heading=bibintoc]

\end{document}
